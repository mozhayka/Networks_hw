\documentclass[a4paper,11pt]{article}
\usepackage{amsmath,amsthm,amssymb}
\usepackage[utf8]{inputenc}
\usepackage[english,russian]{babel}
\usepackage[export]{adjustbox}
\usepackage{graphicx}
\usepackage{pgfplots}
\usepackage{textcomp}

\graphicspath{{pictures/}}
\DeclareGraphicsExtensions{.pdf,.png,.jpg}
\leftskip=-0cm 
\rightskip=-0cm
\voffset = -3cm
\hoffset = -3cm
\textwidth = 550pt
\textheight = 770pt
\pgfplotsset{width=10cm,compat=1.9}


\begin{document}
\Large
HW8
\\
1

Пусть L - потеря пакетов. Тогда средняя пропускная способность равна
$$X = \frac{1.22 \cdot MSS}{RTT \cdot \sqrt{L}}$$

Тогда
$$L = (\frac{1.22 \cdot MSS}{RTT \cdot X})^2$$

Между двумя потерями пакетов отправляется $\frac{1}{L}$ пакетов, то есть 
$$T = \frac{MSS \cdot \frac{1}{L}}{X}  = \frac{RTT^2 \cdot X}{1.22^2 \cdot MSS}$$

Значит T - функция от $X$, то есть от средней пропускной способности
\\
2
\\
a) $$2(\frac{S}{R} + RTT) + 2RTT + 12\frac{S}{R} = 4RTT + 14\frac{S}{R}$$
b) $$3(\frac{S}{R} + RTT) + 2RTT + 8\frac{S}{R}  = 5RTT + 11\frac{S}{R}$$
a) $$(\frac{S}{R} + RTT) + 2RTT + 14\frac{S}{R} = 3RTT + 15\frac{S}{R}$$
\\
3

Пусть было отправлено $n$ пакетов. Тогда $$\frac{W}{2} \cdot (a + 1)^n = W$$ 

то есть $n = \log_{a + 1}2$. Также 
$$S = \frac{W}{2} + \frac{W}{2} \cdot (a + 1) + ... + \frac{W}{2} \cdot (a + 1)^{n - 1} = \frac{W}{2} \cdot \frac{1 - (a + 1)^n}{1 - (a + 1)} = \frac{W}{2} \cdot \frac{1}{a} = \frac{W}{2a}$$

Тогда $$L = \frac{1}{S} = \frac{2a}{W}$$
\end{document}








































