\documentclass[a4paper,11pt]{article}
\usepackage{amsmath,amsthm,amssymb}
\usepackage[utf8]{inputenc}
\usepackage[english,russian]{babel}
\leftskip=-2cm 
\rightskip=-2cm

\begin{document}
\Large
HW2 \\\\
1) Посмотрим, через какое время дойдет последний пакет. Сначала он ждет в очереди $(P - 1) \frac{L}{R}$ времени, а затем за $N \frac{L}{R}$ доходит до приемника. В итоге получаем формулу $(N + P - 1) \frac{L}{R}$ 
\\
2) Пропускная способность будет 200 Кбит/с. Значит времени потребуется 5 Мб / 200 Кбит/с = 40000 / 200 с = 200 секунд
\\
3) Вероятность, что одновременно отправляют $x$ пользователей $$C_{60}^x 0.2^x 0.8^{60 - x}$$. Тогда вероятность, что хотя бы $12$ пользователей, это $$\sum_{12}^{60} C_{60}^x 0.2^x 0.8^{60 - x} \approx 0.55$$
\\
4) Пусть разбиваем на $n$ сегментов. Тогда задержка будет $$(2 + n) \frac{80 + \frac{x}{n}}{R} = \frac{160 + 80n + \frac{2x}{n} + x}{R}$$ 
$x$ и $R$ фиксированы, значит надо минимизировать $$80n + \frac{2x}{n}$$ Так как их произведение фиксировано, то минимум, когда оба слагаемых равны, то есть $$n^2 = \frac{x}{40}$$ $$n = \sqrt{\frac{x}{40}}$$ $$ s = \frac{x}{n} = \sqrt{40x}$$
\end{document}