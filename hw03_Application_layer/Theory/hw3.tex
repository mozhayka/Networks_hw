\documentclass[a4paper,11pt]{article}
\usepackage{amsmath,amsthm,amssymb}
\usepackage[utf8]{inputenc}
\usepackage[english,russian]{babel}
\leftskip=-2cm 
\rightskip=-2cm

\begin{document}
\Large
HW3 \\\\
1) Один пакет на хосте $A$ создается за $$\frac{56 \cdot 8}{128000} = 0,0035 s$$
Время передачи пакета будет $$\frac{56 \cdot 8}{1000000} \approx 0,0004 s$$
Итого $$5 + 0.4 + 3.5 = 8.9 ms$$
\\\\
2) Задержка передачи пакета $$d = \frac{1000}{100} = 10 ms$$
$$N = 10 + 1 = 11$$
$$a \cdot 10 = 11$$
$$a = 0.55$$
То есть средняя скорость $550$ пакетов в секунду
\\\\
3) \\ 
a) Первый пакет прийдет через $$\frac{L}{R_S} + \frac{L}{R_C} + 2 d_{\text{распр}}$$
Второй $$\frac{L}{R_S} + \frac{L}{R_S} + \frac{L}{R_C} +  2 d_{\text{распр}}$$
Тогда разница времени $$\frac{L}{R_S}$$
\\
b) Может, так как второй пакет пройдет по первой линии быстрее, чем первый пройдет полностью по второй. Чтобы очереди не было, надо сделать задержку $T$ такую, что $$T + \frac{L}{R_S} + d_{\text{распр}} \geq \frac{L}{R_C} + d_{\text{распр}}$$ То есть $T \geq \frac{L}{R_C} - \frac{L}{R_S}$
\\\\
4) \\
a) $$\Delta = \frac{850000}{15000000} \approx 0.057 s$$
\\
b) Общее среднее время ответа $$3 + \frac{\Delta}{1 - \Delta \cdot B} = 3 + \frac{0.057}{1 - 0.057 \cdot 16} \approx 3.647 s$$
\\
c) Получение объекта из кеша происходит за $$\frac{850000}{100000000} = 0.0085 s$$
При этом дополнительно с вероятностью $0.4$ запрос идет в интернет, то есть еще $3.647 s$. Тогда в итоге среднее время $$0.0085 + 3.647 \cdot 0.4 = 1.4673 s$$
\end{document}