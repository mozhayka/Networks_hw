\documentclass[a4paper,11pt]{article}
\usepackage{amsmath,amsthm,amssymb}
\usepackage[utf8]{inputenc}
\usepackage[english,russian]{babel}
\usepackage[export]{adjustbox}
\usepackage{graphicx}
\usepackage{pgfplots}
\usepackage{textcomp}

\graphicspath{{pictures/}}
\DeclareGraphicsExtensions{.pdf,.png,.jpg}
\leftskip=-0cm 
\rightskip=-0cm
\voffset = -3cm
\hoffset = -3cm
\textwidth = 550pt
\textheight = 770pt
\pgfplotsset{width=10cm,compat=1.9}


\begin{document}
\Large
HW5 
\\
1
\\

Время распространения 
$$t = \frac{10}{3 \cdot 10^8} = 3.3 \cdot 10^{-8} c$$ 

Общее время, необходимое для получения всех объектов при параллельных непостоянных HTTP-соединениях
$$(\frac{3 \cdot 200}{150} + \frac{100000}{150} + 4 t) + (\frac{3 \cdot 200}{15} + \frac{100000}{15} + 4 t) \approx 7377 c$$

Общее время для постоянных HTTP-соединений
$$(\frac{3 \cdot 200}{150} + \frac{100000}{150} + 4 t) + 10 \cdot (\frac{200}{150} + \frac{100000}{150} + 2 t) \approx 7351 c$$

Разница по времени меньше процента, то есть ускорения почти нет
\\
2
\\

В клиент-серверном взаимодействии сервер передает $F \cdot N$ данных со скоростью $u_s$, кроме случая, когда у клиента скорость приема меньше. То есть при $N = 10$ будет $\frac{F \cdot N}{N \cdot d_i} = \frac{15000}{2} = 7500 c$, при остальных $\frac{F \cdot N}{u_s} = 500 N c$.


\begin{tikzpicture}
\begin{axis}[
    title={Клиент-сервер},
    xlabel={N},
    ylabel={T},
    xmin=0, xmax=1000,
    ymin=0, ymax=500000,
    xtick={10,100, 1000},
    ytick={5000, 50000, 500000},
    legend pos=north west,
]

\addplot[
    color=blue,
    mark=square,
    ]
    coordinates {
    (10,7500)
    (100,50000)
    (1000,500000)
    };
    \legend{}
    
\end{axis}
\end{tikzpicture}

В одноранговом варианте скорость будет $max(u_s + N \cdot u, N \cdot d_i)$. То есть при $N = 10$ или при $u = d_i = 2Мбит/с$ скорость $N \cdot d_i$, иначе скорость $u_s + N \cdot u$. То есть время $\frac{F \cdot N}{N \cdot d_i} = \frac{15000}{2} = 7500 c$ в первом случае и $\frac{F \cdot N}{u_s + N \cdot u}$ во втором

\begin{tikzpicture}
\begin{axis}[
    title={Одноранговая},
    xlabel={N},
    ylabel={T, c},
    xmin=0, xmax=1000,
    ymin=0, ymax=50000,
    xtick={10,100, 1000},
    ytick={7500, 15000, 20000, 25000, 45000},
    legend pos=north west,
]

\addplot[
    color=blue,
    mark=square,
    ]
    coordinates {
    (10,7500)
    (100,7500)
    (1000,7500)
    };
    \addlegendentry{u = 2Мбит/с}
   
\addplot[
    color=red,
    mark=square,
    ]
    coordinates {
    (10,7500)
    (100,15000)
    (1000,20500)
    };
    \addlegendentry{u = 0.7Мбит/с} 
    
\addplot[
    color=green,
    mark=square,
    ]
    coordinates {
    (10,7500)
    (100,25000)
    (1000,45000)
    };
    \addlegendentry{u = 0.3Мбит/с} 
\end{axis}
\end{tikzpicture}


3
\\

а. При передачи каждому со скоростью $\frac{u_s}{N}$

б. При передачи каждому со скоростью $d_{min}$

в. Пусть для каждого клиента скорость $u_i$. 

Тогда $\sum u_i \leq u_s$, то есть $$u_{min} \leq \frac{u_s}{N}$$ и $$u_i \leq d_{min}$$. 

Тогда $$\frac{F}{u_{min}} \geq \frac{NF}{u_s}$$ и $$\frac{F}{u_{min}} \geq \frac{F}{d_{min}}$$

То есть $$\frac{F}{u_{min}} \geq max(\frac{NF}{u_s}, \frac{F}{d_{min}})$$

\end{document}








































