\documentclass[a4paper,11pt]{article}
\usepackage{amsmath,amsthm,amssymb}
\usepackage[utf8]{inputenc}
\usepackage[english,russian]{babel}
\usepackage[export]{adjustbox}
\usepackage{graphicx}
\usepackage{pgfplots}
\usepackage{textcomp}

\graphicspath{{pictures/}}
\DeclareGraphicsExtensions{.pdf,.png,.jpg}
\leftskip=-0cm 
\rightskip=-0cm
\voffset = -3cm
\hoffset = -3cm
\textwidth = 550pt
\textheight = 770pt
\pgfplotsset{width=10cm,compat=1.9}


\begin{document}
\Large
HW13
\\
1

a) $$(Np(1 - p)^{N - 1})' = N (1 - Np) (1 - p)^{N - 2} = 0$$
$$p = \frac{1}{N}$$
При $p = 0$ или $p = 1$ значение изначального выражения будет $0$, в $p = \frac{1}{N}$ больше $0$, значит это и есть максимум

b) $$N \frac{1}{N} (1 - \frac{1}{N})^{N - 1} = (1 - \frac{1}{N})^N \cdot (1 - \frac{1}{N})^{-1} \to \frac{1}{e}$$
\\
2
\\
a) Вероятность передать информацию в кванте 5 $$p(1 - p)^3$$ 

Если учитывать еще вероятность не передать до этого, то надо домножить на $$(1 - p(1 - p)^3)^4$$ 

Итого $$p(1 - p)^3(1 - p(1 - p)^3)^4$$
\\
b) Для всех трех узлов эти вероятности равны, для одного узла вероятность передать информацию в кванте 4 $$p(1 - p)^3$$

Значит вероятность, что хоть кто-то сможет $$3p(1- p)^3$$
\\
c) Вероятность, что передача успешная $$4p(1 - p)^3$$

Вероятность, что 2 передачи до этого были неуспешными $$(1 - 4p(1 - p)^3)^2$$

Итого $$(1 - 4p(1 - p)^3)^2 4p(1 - p)^3$$
\\
d) По задче 1 эффективность $$4p(1 - p)^3$$

При этом максимум достигается при $p = \frac{1}{4}$, тогда значение будет равно $$(1 - \frac{1}{4})^3 = (\frac{3}{4})^3 \approx 0.42$$
\\
3

Время одного опроса 
$$\frac{Q}{R} + d_{\text{опроса}}$$

Cуммарное время всех 
$$N(\frac{Q}{R} + d_{\text{опроса}})$$

Максимальная пропускная способность 
$$\frac{NQ}{N(\frac{Q}{R} + d_{\text{опроса}})} = \frac{Q}{\frac{Q}{R} + d_{\text{опроса}}} = \frac{QR}{Q + R d_{\text{опроса}}}$$
\end{document}








































